% Suitable document classes are ltxdoc and xdpdoc
\documentclass{ltxdoc}
\usepackage[english]{babel}

% Use the titlepage option for setting \author and \date automatically
% Use the authors option in order to base author and date on the last commit
% The rootdir option is an optimization for continuous compilation
% with latexmk's -pvc option
\usepackage[titlepage,authors,rootdir]{gitinfo-lua}

% Decide wether you'd like to use multicols.
% This would only be useful for sections containing a lot of texts, not code
\usepackage{multicol}

\begin{document}
    % Setting the title just before showing it
    \title{My Package Manual%
        % Add a footnote which states package version and date
        \thanks{%
            This manual corresponds to \texttt{my-package}
            version \gitversion\ written on \gitdate.}}%
    \maketitle
    % A summary for this document
    \begin{abstract}
        ...
    \end{abstract}
    % Show the table of contents and macro index in a two-sided column view
    \begin{multicols}{2}
        \tableofcontents \columnbreak \printindex
    \end{multicols}
\end{document}